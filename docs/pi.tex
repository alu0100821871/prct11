\documentclass{beamer}
\usepackage[utf8]{inputenc}
\usepackage{graphicx}

\newtheorem{definicion}{Definición}
\newtheorem{ejemplo}{Ejemplo}

%%%%%%%%%%%%%%%%%%%%%%%%%%%%%%%%%%%%%%%%%%%%%%%%%%%%%%%%%%%%%%%%%%%%%%%%%%%%%%%
\title[Número $\pi$]{El número $\pi$ }
\author[Iciar González Alonso]{Iciar González Alonso}
\date[25-03-2014]{25 de marzo de 2014}
%%%%%%%%%%%%%%%%%%%%%%%%%%%%%%%%%%%%%%%%%%%%%%%%%%%%%%%%%%%%%%%%%%%%%%%%%%%%%%%

\usetheme{Madrid}

\definecolor {pantone254}{RGB}{122,59,122}
\definecolor {pantone3015}{RGB}{0,88,147}
\definecolor {pantone432}{RGB}{56,61,66}
\setbeamercolor*{palette primary}{use=structure, fg=white, bg=pantone254}
\setbeamercolor*{palette secondary}{use=structure, fg=white, bg=pantone3015}
\setbeamercolor*{palette tertiary}{use=structure, fg=white, bg=pantone432}

%%%%%%%%%%%%%%%%%%%%%%%%%%%%%%%%%%%%%%%%%%%%%%%%%%%%%%%%%%%%%%%%%%%%%%%%%%%%%%%
\begin{document}
  
%++++++++++++++++++++++++++++++++++++++++++++++++++++++++++++++++++++++++++++++
\begin{frame}

  \includegraphics[width=0.15\textwidth]{img/ullesc}
  \hspace*{7.0cm}
  \includegraphics[width=0.16\textwidth]{img/fmatesc}
  \titlepage

  \begin{small}
    \begin{center}
     Facultad de Matemáticas \\
     Universidad de La Laguna
    \end{center}
  \end{small}

\end{frame}
%++++++++++++++++++++++++++++++++++++++++++++++++++++++++++++++++++++++++++++++

%++++++++++++++++++++++++++++++++++++++++++++++++++++++++++++++++++++++++++++++
\begin{frame}
  \frametitle{Índice}
  \tableofcontents[pausesections]
\end{frame}
%++++++++++++++++++++++++++++++++++++++++++++++++++++++++++++++++++++++++++++++


\section{Objetivo}


%++++++++++++++++++++++++++++++++++++++++++++++++++++++++++++++++++++++++++++++
\begin{frame}

\frametitle{Objetivo}

\begin{definicion}

El objetivo de esta práctica es entregar una presentación usando \textsc{beamer} que verse sobre el número $\pi$.

\end{definicion}

\end{frame}
%++++++++++++++++++++++++++++++++++++++++++++++++++++++++++++++++++++++++++++++

\section{Número pi}

\subsection{Definición}
%++++++++++++++++++++++++++++++++++++++++++++++++++++++++++++++++++++++++++++++
\begin{frame}
\frametitle{Definición}
$\pi$ es la relación entre la longitud de una circunferencia y su diámetro, en ggeometría euclidiana. Es un número irracional y una de las constantes matemáticas más importantes. 
El valor numérico del número $\pi$, expresado con sus primeras cincuenta cifras decimales es el siguiente: 
\begin{center}
$\pi$ ~ 3.14159265358979311599796346854418516159057617187500
\end{center}

El valor de $\pi$ se ha obtenido con diversas aproximaciones a lo largo de la historia. Junto con el número e, es una de las constantes matemáticas más utilizadas.

\end{frame}
%++++++++++++++++++++++++++++++++++++++++++++++++++++++++++++++++++++++++++++++

\subsection{Notación}

%++++++++++++++++++++++++++++++++++++++++++++++++++++++++++++++++++++++++++++++
\begin{frame}
\frametitle{Notación}

\begin{definicion}
  La notación con la letra griega $\pi$ proviene de las palabras de origen griego "periferia" y "perímetro" de un círculo.
Esta notación fue utilizada por primera vez por Willian Oughtred (1574-1660) y aunque su uso fue propuesto por el matemático galés Willian Jones (1675-1749) fue el matemático Leonhard Euler con su obra "Introduccion al cálculo infinitesimal", de 1748 quien la popularizó. 
Anteriormente había sido conocida como "Constante de Arquímedes" y como "constante de Ludolph" (en honor al matemático Ludolph van Ceulen).\cite{URL:HTTP}
\end{definicion}
\end{frame}
%++++++++++++++++++++++++++++++++++++++++++++++++++++++++++++++++++++++++++++++

\section{Fórmulas}
\subsection{Primera fórmula}
%++++++++++++++++++++++++++++++++++++++++++++++++++++++++++++++++++++++++++++++
\begin{frame}
\frametitle{Primera fórmula}
$$\int_{0}^{1} \! \frac{4}{1+x^2}\, dx = 4(atan(1) -atan(0)) = \pi $$

\end{frame}

\subsection{Segunda fórmula}
%++++++++++++++++++++++++++++++++++++++++++++++++++++++++++++++++++++++++++++++
\begin{frame}
\frametitle{Segunda fórmula}

\[x=\frac{a_2 x ^ 2 + a_1 x + a_0}{1+2z ^ 3}\]
\end{frame}


\subsection{Tercera fórmula}
%++++++++++++++++++++++++++++++++++++++++++++++++++++++++++++++++++++++++++++++
\begin{frame}
\frametitle{Tercera fórmula}

\[\quad x+y ^ {2n+2}=\sqrt{b ^ 2-4ac}\]
\end{frame}


\subsection{Cuarta fórmula}
%++++++++++++++++++++++++++++++++++++++++++++++++++++++++++++++++++++++++++++++
\begin{frame}
\frametitle{Cuarta fórmula}
\[ S_n=a_1+\cdots + a_n = \sum_{i=1} ^ n a_i \]

\end{frame}


\subsection{Quinta fórmula}
%++++++++++++++++++++++++++++++++++++++++++++++++++++++++++++++++++++++++++++++
\begin{frame}
\frametitle{Quinta fórmula}
\[\int_{x=0} ^ {\infty} x\, \text{e} ^ {-x ^ 2} \text{d}x=\frac{1}{2}, \quad\text{e} ^ {i\pi}+1=0\]

\end{frame}


%++++++++++++++++++++++++++++++++++++++++++++++++++++++++++++++++++++++++++++++

 

\section{Historia}

\subsection{Evolución}
%++++++++++++++++++++++++++++++++++++++++++++++++++++++++++++++++++++++++++++++
\begin{frame}
\frametitle{Evolución}

La primera referencia que se conoce actualmente de $\pi$ es aproximadamente del año 1650 a.C. en el Papiro de Ahmes. En este domcumento estaban contenidos numerosos problemas matemáticos básicos, fracciones, cálculo de áreas y volúmenes, ecuaciones, progresiones, trigonometría,... 
El valor que se le da a $\pi$ es 2**8/3**4 ~ 3,1605

Una vez situados en la época de la informática, uno de los métodos para comprobar la eficacia de las máquinas era utilizarlas para calcular decimales de $\pi$. En el año 1949 una computadora ENIAC fue capaz de calcular 2037 decimales en 70 horas, en 1966 un IBM 7030 llegó a 250.000 cifras decimales en 8 horas y 23 minutos y ya en el siglo XXI, en el año 2004, un superordenador Hitachi estuvo trabajando 500 horas para calcular 1,3511 billones de lugares decimales.
\end{frame}
%++++++++++++++++++++++++++++++++++++++++++++++++++++++++++++++++++++++++++++++

\subsection{Cálculo}

%++++++++++++++++++++++++++++++++++++++++++++++++++++++++++++++++++++++++++++++
\begin{frame}
\frametitle{Cálculo}

\begin{definition}

Desde el Antiguo Egipto hasta la Matemática china, india o islámica, pasando por Mesopotamia e incluso referencias bíblicas, las aproximaciones de este número eran bastante rudimentarias. 
A partir del siglo XII y fundamentalmente durante el Renacimiento Europeo y la Época moderna (pre-computacional) estos cálculos se fueron mejorando.
Pero no fue hasta la llegada de los computadores cuando se consiguió calcular un número impensable en siglos pasados de cifras decimales, y a partir del siglo XX su precisión ha ido en aumento, así como va disminuyendo el tiempo que tarda el ordenador en calcular estos decimales.
\end{definition}

\end{frame}
%++++++++++++++++++++++++++++++++++++++++++++++++++++++++++++++++++++++++++++++

\begin{frame}

\frametitle{Items}

\begin{block}{Ejemplo}
  \begin{itemize}
  \item
  Este es el primer item
  \pause

  \item
  Este es el segundo item

  \end{itemize}
\end{block}

\end{frame}



\section{Bibliografía}
%++++++++++++++++++++++++++++++++++++++++++++++++++++++++++++++++++++++++++++++
\begin{frame}
  \frametitle{Bibliografía}

  \begin{thebibliography}{10}

    \beamertemplatebookbibitems
    \bibitem[Plan Estudios, 2011]{plan}
    Documento de verificación del grado.
    (2011)

    \beamertemplatebookbibitems
    \bibitem[Guía Docente, 2013]{guia}
    Guía docente.
    (2013)
    {\small $http://eguia.ull.es/matematicas/query.php?codigo=299341201$}

    \beamertemplatebookbibitems
    \bibitem[URL: CTAN]{latex}
    CTAN. {\small $http://www.ctan.org/$}

  \end{thebibliography}
\end{frame}

%++++++++++++++++++++++++++++++++++++++++++++++++++++++++++++++++++++++++++++++
\end{document}